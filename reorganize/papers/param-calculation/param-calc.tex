\documentclass[10pt]{article} % Font size - 10pt, 11pt or 12pt

\nonstopmode

\usepackage[hmargin=1.25cm, vmargin=1.5cm]{geometry} % Document margins

\usepackage{amsmath}

\usepackage[usenames,dvipsnames]{xcolor} % Allows the definition of hex colors

% Fonts and tweaks for XeLaTeX
\usepackage{fontspec,xltxtra,xunicode}
\defaultfontfeatures{Mapping=tex-text}
%\setmonofont[Scale=MatchLowercase]{Andale Mono}

% Colors for links, text and headings
\usepackage{hyperref}
\definecolor{linkcolor}{HTML}{506266} % Blue-gray color for links
\definecolor{shade}{HTML}{F5DD9D} % Peach color for the contact information box
\definecolor{text1}{HTML}{2b2b2b} % Main document font color, off-black
\definecolor{headings}{HTML}{701112} % Dark red color for headings
% Other color palettes: shade=B9D7D9 and linkcolor=A40000; shade=D4D7FE and linkcolor=FF0080

\hypersetup{colorlinks,breaklinks, urlcolor=linkcolor, linkcolor=linkcolor} % Set up links and colors

\usepackage{fancyhdr}
\usepackage{amssymb}
\pagestyle{fancy}
\fancyhf{}
% Headers and footers can be added with the \lhead{} \rhead{} \lfoot{} \rfoot{} commands
% Example footer:
%\rfoot{\color{headings} {\sffamily Last update: \today}. Typeset with Xe\LaTeX}

\renewcommand{\headrulewidth}{0pt} % Get rid of the default rule in the header

\usepackage{titlesec} % Allows creating custom \section's

% Format of the section titles
\titleformat{\section}{\color{headings}
\scshape\Large\raggedright}{}{0em}{}[\color{black}\titlerule]

\title{Electromagnetism Assignment One}
\author{Elliott Capek}
\titlespacing{\section}{0pt}{0pt}{5pt} % Spacing around titles

\begin{document}

\maketitle{}

\subsection{Calculating one/bothbound times of experimental dynein}
In order to fit parameters, a goodness metric for a simulation is needed. Time spent in onebound and bothbound states are good metrics. These values are calculated here.\\

\newcommand\tbb{\left<t_{bb}\right>}
\newcommand\tob{\left<t_{ob}\right>}
\newcommand\tstep{\left<t_{step}\right>}
\newcommand\tproc{\left<t_{processivity}\right>}
\newcommand\kub{k_{ub}}
\newcommand\kb{k_{b}}
\newcommand\ko{k_{dis}}

\begin{itemize}
\item $\tbb$: Time in bothbound per step
\item $\tob$: Time in onebound per step
\item $\tstep$: Time of a single step. $\tstep=\tbb+\tob$.
\item $P_{bb}$: Probability of being bothbound per unit time. $P_{bb} = \frac{\tbb}{\tstep}$
\item $P_{ob}$: Probability of being onebound per unit time. $P_{ob} = \frac{\tob}{\tstep}$
\item $\kub$: Rate of unbinding per unit time while in bothbound. $\kub = \tbb^{-1}$
\item $\kb$: Rate of binding per unit time while in onebound. $\kb = \tob^{-1}$
\item $\ko$: Rate of unbinding per unit time while in onebound. $\ko = P_{ob}\kub$
\item $\tproc$: Time bound to microtubule before dissociation. step length / velocity $= \ko^{-1}$
\end{itemize}

$\tbb$ and $\tob$ can thus be defined in terms of observables $\tstep$ and $\tproc$:

\begin{align*}
  \ko &= P_{ob}\kub = \frac{\tob}{\tstep\tbb}\\
  &= \frac{\tstep-\tbb}{\tstep\tbb}\\
  &= \frac{\tstep-\tbb}{\tstep\tbb}\\
  \tproc &= \frac{\tstep\tbb}{\tstep-\tbb}\\
  \tbb &= \frac{\tproc\tstep}{\tproc+\tstep}\\
  \tob &= \tstep-\tbb\\
  &= \tstep - \frac{\tstep\tproc}{\tstep+\tproc}\\
  &= \frac{\tstep^2}{\tstep+\tproc}\\
  \frac\tob\tbb &= \frac{\tstep}{\tproc}
\end{align*}

Thus, by the above equations and data from various sources, estimates on $\tbb$ and $\tob$ can be made.\\

\begin{figure}[h]
  \centering
  \begin{tabular}{| c | c | c | c | c | c | c |}
    \hline
    Source & Velocity & $\left<L_{step}\right>$ & $\tstep$ & $\tproc$ & $\tbb$ & $\tob$ \\ \hline
    Weihong 2012 & 134nm/s & 8nm & 0.06s & 7.9s & 0.0595s & 4.52e-4\\ \hline
    Higuchi 2006 & 800nm/s & 8nm & 0.01s &  7.9? &  9.9e-3 & 1.2e-5\\ \hline
  \end{tabular}
  \caption{Experimental values for fitting.}{Table calculating estimates for $\tbb$ and $\tob$.}
  \label{table:time-parameter-table}
\end{figure}


\end{document}
