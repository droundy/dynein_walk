We have developed a Brownian dynamics simulation which effectively models the locomotion of the dynein motor protein. By manipulating our binding and unbinding rate parameters together with the spring constants for each domain, we are able to reproduce step length and velocity distributions as measured experimentally. Furthermore, directed motion towards the same end of the microtubule was achieved across all random number seeds without the need to introduce a separate model for the structure of the microtubule. 

In an effort to corroborate the results of research on the tension dependence of dynein stepping, we introduced an angle dependent unbinding rate which favors lagging foot steps at large inter-domain separations. Simulations with this unbinding model indicate that the one bound and both bound configurations are largely uncorrelated. In other words, the initial configuration of dynein does factor in to which domain steps, but it does not correlate with the final state after the next binding event. This unexpected result is contrary to experiment and has led us to begin developing a new simulation for which we only dynamically simulate the one bound state and use Monte Carlo methods to analyze distributions of possible both bound states. We anticipate that this new simulation will be significantly easier to implement and should lead to faster simulations. 


The flexible design and fast computation time relative to other simulation methods mean that our model can be easily applied to test other experimental hypotheses. For example, a popular mode of research is to use an optical tweezer to test how much force a dynein generates when pulling cargo. Our simulation can easily accommodate additional forces and will allow us to examine how the motion deviates from the case of a free dynein on a microtubule. Future work may also integrate a second dimension into the dynamics enabling us to simulate how dynein navigates around obstacles on it's 3 dimensional surface. By making small additions and adjustments like this, we believe that our simulation will help shed light on exactly how dynein moves throughout the cell. 