\section{Optimized Parameters}\label{sec:Params}
The set of parameters that best fit Yildiz's observed stepping correlation is shown below in Table (\ref{tab:params}).

\begin{table}[H]
  \centering
  \begin{tabular}{|l | r | r | r|}
  	\hline
    Param & Model & Experimental & Source \\
    \hline
    $c_b$ & $\cb \Delta G_{ATP}$ &  & \\
    $c_m$ & $\cm \Delta G_{ATP}$ &  & \\
    $c_t$ & $\ct \Delta G_{ATP}$ &  & \\
    $k_a$ & $\kstk s^{-1}$&  & \\
    $k_b$ & $\kb s^{-1}$&  & \\
    $k_{ub}$ & $\kub s^{-1}$ & & \\
    $C$ & $\cexp$ & & \\
    $L_s$ & $\ls nm$ & $21nm$ & \cite{Burgess2003, 3vkh-cite, carter-paper}\\
    $L_t$ & $\lt nm$ & $23nm$ & \cite{Burgess2003, 3vkh-cite, carter-paper}\\
    $\theta_b$ & $\eqb$ &  120 & \cite{leschziner} \\
    $\theta_m^{\mbox{pre}}$ & $\eqmpre$ &  197 & \cite{Burgess2003}\\
    $\theta_m^{\mbox{post}}$ & $\eqmpost$ & 242 & \cite{Burgess2003}\\
    $\theta_t$ & $\eqt$ &  & \\
    $R_t$ & $\radiust nm$ & $4nm$ & \cite{Burgess2003}\\
    $R_m$ & $\radiusm nm$ & $5.5nm$ & \cite{Burgess2003}\\
    $R_b$ & $\radiusb nm$ & $1.25nm$ & \cite{Burgess2003}\\
    \hline
  \end{tabular}
  \caption{Optimized model parameters for Figures (\ref{fig:DataFitYildiz}-\ref{fig:Trajectory}).}
  \label{tab:params}
\end{table}

After many trials, this combination of parameters best demonstrated the interstep correlation observed from Yildiz's experiment, while still preserving dynein's forward directionality. The two main stepping statistics used to compare with experiment are shown below, where the affinity transition rate ($k_a$) and binding rate ($k_b$) heavily influenced these statistics.  

\begin{figure}[H]
	\centering
	\includegraphics[width=0.8\textwidth]{/mc_plots/u_final_disp_probability_distribution_2.0_5.50e+10_1.00e+08_0.0_1.0_1.0_120.0_197.0_242.0_-0.5}
	\includegraphics[width=17cm, height = 9.5cm]{/mc_plots/u_step_length_1d_probability_density_2.0_5.50e+10_1.00e+08_0.0_1.0_1.0_120.0_197.0_242.0_-0.5}
	\caption[Final Displacement Probability Distribution]{\textbf{Stepping Probability Distribution Plots.} \textit{Top:} Two dimensional heatmap of binding domain displacement before and after a step. Linear regression of the probability distribution indicates a dependence between the final displacement ($x_f$) and initial displacement ($x_i$). \textit{Bottom:} One dimensional probability density of step length (step length = $x_f-x_i$). Model compared with experimental figures from \citep{Dewitt2012}.} 
	\label{fig:DataFitYildiz}
\end{figure}
\newpage
The first plot in Figure (\ref{fig:DataFitYildiz}) displays the joint probability distribution of having a specific initial displacement and final displacement ($p(x_i,x_f)$). The linear regression from Yildiz's experiment indicates a functionality between the two with an average final displacement of 9.1 nm, shown from the y-intercept, and a slope of 0.6. The model produced steps with an average final displacement of 6.28 nm and a slope of 0.501. Despite the $16.5\%$ percent error in the slope, the model dynein was able to reproduce Yildiz's interhead correlation only if dynein prefers fast stutter steps with step lengths close to 0 nm. This is apparent from the diagonal shaded region with a slope of 1, indicating a step where $x_f\approx x_i$.


The bottom plot in Figure (\ref{fig:DataFitYildiz}) displays the one-dimensional probability density of step lengths, defined as final displacement ($x_f$) minus initial displacement ($x_i$). The accumulative density from the diagonal shaded region is evident in this plot due to the peak at 0 nm. However, experiment was unable to detect 0 nm steps and is shown from the dip in Experiment Fig 3A (small density at 0 nm were lateral steps on the off-axis with little change on the on-axis). Our model suggests that dynein can still achieve forward directed motion with more positive steps than negatives ones but prefers short stuttered steps with fast one-bound times. A distribution of said one-bound times is shown below in Figure (\ref{fig:OBtime}) 

\begin{figure}[H]
	\centering
	\includegraphics[width=14cm, height = 10cm]{../../plots/mc_plots/u_ob_time_probability_density_2.0_5.50e+10_1.00e+08_0.0_1.0_1.0_120.0_197.0_242.0_-0.5}
	\caption[One-bound times]{\textbf{One-bound times.} Distribution of time spent in the one-bound state from either a trailing or leading step for an initial distance $L=16$. Average affinity time labeled at $\langle t_a \rangle \approx 0.01 \mu s$, and probability outside the time range is $P(t>0.5\mu s) \approx 0.0068$.}
	\label{fig:OBtime}
\end{figure}

As the distribution of times in Figure (\ref{fig:OBtime}) indicates, the model dynein frequently spends $<$ 0.1 $\mu s$ in the one-bound state, with the dynein most likely to rebind immediately after the affinity transitions. This demonstrates the importance of a buffer time when the dynein unbinds to avoid instantaneous rebinding while still achieving large positive step lengths. A deeper analysis of this will be discussed in the Limiting Cases section (Section \ref{sec:LimitingCase}).

We also fit the pre-exponential unbinding factor ($C$) and unbinding rate ($k_{ub}$) based on the following experimental plots.

\begin{figure}[H]
	\centering
	\includegraphics[width=15.5cm, height = 6cm]{../../plots/mc_plots/prob_lagging_vs_init_L_-0.5}
	\includegraphics[width=15cm, height = 10cm]{../../plots/mc_plots/u_trajectory_plot_2.0_5.50e+10_1.00e+08_0.0_1.0_1.0_120.0_197.0_242.0_-0.5}
	\caption[Fitting Bothbound Parameters]{\textbf{Fitting Bothbound Parameters.} \textit{Top: }Likelihood of a trailing step vs. the initial binding domain distance ($P_{trail.}(L)$ from Equation (\ref{eqn:ProbTrail})). \textit{Bottom: }Trajectory of binding domains over 24 seconds of walking. Experimental data from \cite{Dewitt2012}.}
	\label{fig:BBPlots}
\end{figure}

By measuring the unbinding rates from the ensemble of both-bound configurations, we were able to accurately reproduce the experiments unbinding probability for the trailing leg as a function of the initial binding domain separation (L). This result forced the model to prefer trailing steps for large initial distances and influenced forward directed stepping from the trailing leg's powerstroke. The model dynein also replicated experiment's velocities over a long walk by modifying the unbinding rate ($k_{ub}$) and comparing the slopes of the binding position trajectories. Both the experiment and model were able to record dynein's average velocity to be $v\approx25$ nm/s. Additionally, the bottom plot in Figure (\ref{fig:BBPlots}) displays the model's ability to replicate dynein's ``drunken" walk, with steps that vary forward and backwards. 

Based on the resemblence with experimental observations, we are fairly confident with the model's ability to replicate the stepping patterns of real dynein. In our spring model, we assumed each domain to act as springs to influence the powerstroke. However, the simulations displayed more consistent forward stepping when we turned off the binding domain spring and only allowed the motor and tail to have non-zero spring constants. That is, $c_b=0$ while $c_m=c_t=1$ from Table (\ref{tab:params}). This adjustment displays the flexibility within the model and the potential to fine-tune paramters to fit different sets of data. 

\newpage
\section{Limiting Case}\label{sec:LimitingCase} 

As mentioned previously, the model dynein had to take very rapid steps in order to closely generate Yildiz's stepping correlation. To lessen the error and achieve a slope closer to Yildiz's value of 0.6, the binding rate ($k_b$) and affinity transition rate ($k_a$) must both increase significantly. This led to instantaneous rebinding, as we investigated the limiting case of $k_a\gg k_b$ (affinity transition time $t_a\to 0$) and is shown in Figure (\ref{fig:DataFitYildiz99})

\begin{figure}[H]
	\begin{minipage}[b]{0.5\textwidth}
	\includegraphics[width=8.5cm, height=6.5cm]{/mc_plots/u_final_disp_probability_distribution_2.0_5.50e+10_9.00e+99_0.0_1.0_1.0_120.0_197.0_242.0_-0.5}
	\end{minipage}
	\begin{minipage}[b]{0.6\textwidth}		
	\includegraphics[width=9.2cm, height = 6cm]{/mc_plots/u_step_length_1d_probability_density_2.0_5.50e+10_9.00e+99_0.0_1.0_1.0_120.0_197.0_242.0_-0.5}
	\end{minipage}	
	\begin{center}
	\includegraphics[width=12cm, height = 7.5cm]{../../plots/mc_plots/u_ob_time_probability_density_2.0_5.50e+10_9.00e+99_0.0_1.0_1.0_120.0_197.0_242.0_-0.5}
	\end{center}	
	\caption[Limiting Case]{\textbf{Limiting Case for Instant Rebinding.} Each plot closely resembled the plots shown before but with the limiting case of $k_a\gg k_b$. The steps exhibited instantaneous rebinding, where the lifted domain landed at the same position immediately after unbinding off of the microtubule ($x_f\approx x_i$). The large majority of one-bound times were recorded within the first time-step where $dt<5$ ns.} 
	\label{fig:DataFitYildiz99}
\end{figure}

When fitting to Yildiz's slope, we avoided instant rebinding by also prioritizing the large positive step lengths Yildiz observed in Experiment Fig 3A (on Figure \ref{fig:DataFitYildiz}). The binding and affinity transition rates can be better optimizied to fit the larger step lengths (we classified ``large'' steps as $> 40$ nm), but doing so would decrease the correlation between the displacements and thus, the value for the slope. This caused an optimization problem of whether to fit according to the slope or the step lengths. We believed that the parameters shown in Table (\ref{tab:params}) best demonstrated the functionality of the displacements from the slope, while maintaining dynein's forward directionality with a similar step length trend towards large steps.

%\begin{figure}[H]
%	\centering
%	\includegraphics[width=0.7\textwidth]{/mc_plots/u_final_disp_probability_distribution_2.0_5.50e+09_1.00e+08_0.0_1.0_1.0_120.0_197.0_242.0_-0.5}
%	\includegraphics[width=0.65\textwidth]{/mc_plots/u_step_length_1d_probability_density_2.0_5.50e+09_1.00e+08_0.0_1.0_1.0_120.0_197.0_242.0_-0.5}
%	\caption[Final Displacement Probability Distribution]{\textbf{Stepping Probability Distribution Plots.} \textit{Top:} Two dimensional heatmap of binding domain displacement before and after a stepping cycle. Linear regression of the probability distribution indicates a dependence between the final displacement of the step and its initial displacement. \textit{Bottom:} One dimensional probability density of step length. Step length defined as final displacement minus the initial displacement. Model compared to two experimental figures of stepping patterns from \citep{Dewitt2012}.} 
%	\label{fig:DataFitYildiz09}
%\end{figure}
%
%\begin{figure}[H]
%	\centering
%	\includegraphics[width=0.8\columnwidth]{../../plots/mc_plots/u_ob_time_probability_density_2.0_5.50e+09_1.00e+08_0.0_1.0_1.0_120.0_197.0_242.0_-0.5}
%	\caption[One-bound times]{\textbf{One-bound times.} Distribution of time spent in the one-bound state from either a trailing or leading step for an initial distance $L=16$. Average affinity time labeled at $\langle t_a \rangle \approx 0.01 \mu s$, and probability outside the time range is $P(t>0.5\mu s) \approx 0.0066$.}
%	\label{fig:OBtime09}
%\end{figure}


%\section{Agree with Experiment?}
%\textit{Whole point of parameter fitting was to agree with Yildiz and/or other experimental data. Did we do a good job? }
%
%\textit{Need to eventually bring up the issue of being able to fit both the 2d hist linear regression and step length plot: If we want to fit linear regression, we must have short steps, but if we have short steps, the step length plot will not match at steps $>$ 20nm. Hard to agree with experiment when our dynein is forced to take short steps. Mention time too.}
