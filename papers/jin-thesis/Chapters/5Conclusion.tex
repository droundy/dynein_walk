
%\textit{What do our results indicate? Do we think our model correctly represents real dynein and its stepping patterns?}

We successfully incorporated a Monte Carlo algorithm when dynamically simulating dynein's unique stepping mechanism. By modelling dynein as a two-dimensional particle-rod system, we separated the conformations of the mechanochemical cycle into a both-bound and one-bound state and emulated the powerstroke with springs in each domain. With various degrees of freedom controlling the properties of a step, the model was able to flexibly optimize according to experimental results and provide further conclusions for correlated stepping. More specifically, we replicated experimentalist Ahmet Yildiz's unique result of an observered stepping correlation between the initial domain separation and final displacement with a 16.5\% error.

Despite experiments inability to observe step lengths close to 0 nm, we conclude that dynein must take short and fast steps in order to achieve strong stepping dependence. After many tests with different combinations of parameters, the set of parameters shown in Table (\ref{tab:params}) demonstrated Yildiz's stepping correlation while still preserving directed motion with similar positive step length trends. Dynein also preferred larger positive steps when only the motor and tail domains acted as springs. Furthermore, the model achieved dynein's stepping tendencies with biased trailing steps and identical walking trajectories, all while being more computationally efficient and faster than previous simulations. 

If time provides, the flexbility of the model allows for a more sufficient optimization process that can minimize the parameter space when replicating experiment. Further work can apply this to other experimental hypotheses regarding dynein's stepping pattern and provide further support for new observations. As mentioned earlier, most simulations model the ATP transitions during the mechanochemical cycle, while this model focuses on the dynamics within the step. Thus, a combined model that incorporates both features may describe natural dynein perfectly and bring us one step closer to engineering a reliable prevention for cell failure. 

