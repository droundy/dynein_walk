\documentclass[10pt]{article} % Font size - 10pt, 11pt or 12pt

\usepackage[hmargin=1.25cm, vmargin=1.5cm]{geometry} % Document margins

\usepackage{marvosym} % Required for symbols in the colored box

\usepackage[usenames,dvipsnames]{xcolor} % Allows the definition of hex colors

% Fonts and tweaks for XeLaTeX
\usepackage{fontspec,xltxtra,xunicode}
\defaultfontfeatures{Mapping=tex-text}
%\setmonofont[Scale=MatchLowercase]{Andale Mono}

% Colors for links, text and headings
\usepackage{hyperref}
\definecolor{linkcolor}{HTML}{506266} % Blue-gray color for links
\definecolor{shade}{HTML}{F5DD9D} % Peach color for the contact information box
\definecolor{text1}{HTML}{2b2b2b} % Main document font color, off-black
\definecolor{headings}{HTML}{701112} % Dark red color for headings
% Other color palettes: shade=B9D7D9 and linkcolor=A40000; shade=D4D7FE and linkcolor=FF0080

\hypersetup{colorlinks,breaklinks, urlcolor=linkcolor, linkcolor=linkcolor} % Set up links and colors

\usepackage{fancyhdr}
\usepackage{amsmath}
\usepackage{physics}
\usepackage{amssymb}
\pagestyle{fancy}
\fancyhf{}
% Headers and footers can be added with the \lhead{} \rhead{} \lfoot{} \rfoot{} commands
% Example footer:
%\rfoot{\color{headings} {\sffamily Last update: \today}. Typeset with Xe\LaTeX}

\renewcommand{\headrulewidth}{0pt} % Get rid of the default rule in the header

\usepackage{titlesec} % Allows creating custom \section's

% Format of the section titles
\titleformat{\section}{\color{headings}
\scshape\Large\raggedright}{}{0em}{}[\color{black}\titlerule]

\title{Dynein project log}
\author{Elliott Capek}
\titlespacing{\section}{0pt}{0pt}{5pt} % Spacing around titles

\begin{document}

\maketitle{}

\section{Equipartition Theorem}
The Equipartition Theorem (ET) predicts the average energy per degree of freedom of a system with energy which depends on the square of each degree of freedom. Mathematically is is expressed as:

\begin{align*}
  Z &= \int_{-\infty}^{\infty} e^{-\beta Ax^2}dx = \sqrt{\frac{\pi}{\beta A}}\\
  <E> &= - \frac{d\ln Z}{d\beta} = -\left(\sqrt{\frac{\beta A}{\pi}}\right)\left(\frac{\sqrt{\beta A}}{2\sqrt{\pi}}\right)\frac{-\pi}{A\beta^2}\\
  &= \frac{1}{2}k_BT\\
\end{align*}

where Z is the partition function and x a degree of freedom.\\

For our model, each domain feels three separate forces:

\textbf{Brownian forces}: Random forces due to water particle collisions. Gaussian-sampled with a
variance $\sqrt{\frac{\gamma k_BT}{dt}}$. Brownian forces are somehow necessary for the ET energy
to be realized.\\

\textbf{Conformational forces}: Spring forces from the molecule deviating from its lowest-energy
conformation. The energy from these forces is quadratic with respect to angle, and so each free
angle should contribute to the average energy by ET.\\

\textbf{Internal forces}. Tension forces (constraint forces) felt between domains which keep
the protein rigid. These are not quadratic, so it is unclear how they contribute to energy!\\

\subsection{Onebound case}
The conformational forces are:\\
\begin{align*}
  |\vec{F_{ub}}| &\approx \frac{c_m\theta_{um}}{L_s}\\
  |\vec{F_{um}}| &\approx \frac{c_m\theta_{um}}{L_s} + \frac{c_m\theta_{um}}{L_t} + \frac{c_t\theta_{t}}{L_t}\\
  |\vec{F_{t}}| &\approx \frac{c_m\theta_{um}}{L_t} + 2\frac{c_t\theta_t}{L_t} + \frac{c_m\theta_{bm}}{L_t}\\
  |\vec{F_{bm}}| &\approx \frac{c_t\theta_t}{L_t} + \frac{c_m\theta_{bm}}{L_t} + \frac{c_m\theta_{bm}}{L_s} + \frac{c_b\theta_{bb}}{L_s}\\
  |\vec{F_{bb}}| &\approx \frac{c_m\theta_{bm}}{L_s} + \frac{c_b\theta_{bb}}{L_s}\\
\end{align*}

From this we can come up with rough order-of-magnitude estimates of the energy of each domain is:

\begin{align*}
  PE_{ub} &\approx \frac{c_m}{L_s}\theta_{um}^2 + \frac{c_m}{L_t}\theta_{um}^2\\
  PE_{t} &\approx \frac{c_t}{L_t}\theta_{t}^2\\
  PE_{bm} &\approx \frac{c_m}{L_s}\theta_{bm}^2 + \frac{c_m}{L_t}\theta_{bm}^2\\
  PE_{bb} &\approx \frac{c_m}{L_s}\theta_{um}^2\\
\end{align*}

From this we can see that the conformational PE of each domain depends on different combinations of
stalk/tail length and spring constants.

\textbf{When is equipartition behaved?}
It seems like ET should be behaved when we have an even balance of Brownian and conformational forces:

\begin{align*}
  F_{conf} &\approx F_{Brownian}\\
  \frac{c}{L}\theta &\approx \sqrt{\frac{k_BT\gamma}{dt}}\\
\end{align*}

Is this reflected in our simulation? Well take a look at %% Fig ($\ref{fig:latent-heat}$)
...

%% \begin{figure}[h!]
%%   \centering
%%   \includegraphics[width=0.5\textwidth]{../figures/ob_equipartition_vs_force_ratio.pdf}
%%   \caption{Onebound conformational energy follows equipartition best when the ratio of conformational
%%   forces to Brownian forces is one.}
%%   \label{fig:equipartition_vs_force_ratio}
%% \end{figure}

\section{Brownian motion}
\subsection{Derivation} \label{sec:bmotion-derivation}
We can derive the equation for the position of a particle using the simple Newtonian mechanics. We
start with the general $2^{nd}$ law equation for a particle with random forces and drag (the
Langevin equation). We then take the time-averages of all variables (using expectation value signs)
to eliminate the effect of random Brownian forces. After some algebra we arrive at a differential
equation for squared position. We solve it and simplify its expression for limits when $t >> \tau$
and $t << \tau$.\\

\begin{align*}
  \frac{d\vec{v}}{dt} &= \frac{1}{m}\left(\frac{-mv}{\tau} + R\right)\\
  \frac{d\vec{v}}{dt} &= \frac{-v}{\tau} + \frac{R}{m}\\
  <\vec{r} \cdot \frac{d\vec{v}}{dt} > &+ \frac{1}{\tau}<\vec{r} \cdot \vec{v}> = 0\\
  \frac{1}{2}\frac{d^2}{dt^2}(r^2) &+ \frac{1}{2\tau}\frac{d}{dt}(r^2) - v^2 = 0 \hspace{1cm} \Bigg(\mbox{b/c }\frac{d}{dt}\left(\vec{r} \cdot \vec{v}\right) = v^2 + \left(\vec{r} \cdot \frac{d\vec{v}}{dt}\right)
  \mbox{ and }
  \frac{d}{dt}\left(\vec{r} \cdot \vec{r}\right) = 2\left(\vec{r} \cdot \vec{v}\right) \Bigg)\\
  \frac{d^2}{dt^2}(<r^2>) + &\frac{1}{\tau}\frac{d}{dt}(<r^2>) = 2<v^2> = \frac{6k_BT}{m} \hspace{1cm}\left(\mbox{Equipartition Theorem}\right)\\
  <r^2> &= \frac{6k_BT\tau^2}{m}\left(e^{-t/\tau}-1+\frac{t}{\tau}\right)\\
\end{align*}

so...

\begin{align*}
  \left(t << \tau \right) \rightarrow <r^2> = \frac{3k_BT}{m}t^2 \hspace{2cm}\mbox{free particle motion}\\
  \left(t >> \tau \right) \rightarrow <r^2> = \frac{6k_BT\tau}{m}t \hspace{2cm}\mbox{motion proportional to }\sqrt{t}\\
\end{align*}

\textbf{Averaging over time?}
In the above derivation we use time averaging to eliminate the effect of Brownian forces. This is
essentially just discussing the motion of a particle on a timescale where Brownian forces go to
zero. The variance of the Brownian force R is:

\begin{align*}
  \sigma_R^2 &= \sqrt{\frac{2k_BTm}{\tau dt}}\\
\end{align*}

(TODO: further justify the dt in the variance)\\
Thus over larger timescales (bigger dt's) the Brownian force is smaller. When time averaging, all
we are saying is that our dt is large enough that Brownian forces are negligible.\\

An interesting observation is that the motion equation for a particle feeling Brownian forces is
identical to the equation for a macroscopic object feeling drag forces, which makes sense, since
to solve the Brownian motion equation we just eliminate R.\\

\section{Brownian dynamics}
We use Brownian dynamics as our equation of motion. BD is just the special case of Langevin dynamics

\begin{align*}
  m\frac{d\vec{v}}{dt} &= -\frac{m}{\tau}\vec{v} + \vec{F} + \vec{R}\\
\end{align*}

where $m\rightarrow0$:

\begin{align*}
  \vec{v} &= \frac{\tau}{m}\vec{F} + \frac{\tau}{m}\vec{R}\\
\end{align*}

where m is the mass, $\tau$ the time constant, $\vec{F}$ a generic force and $\vec{R}$ the Brownian
force. Mass ``going to zero'' means we are discussing particles where $m << $ any other variable.
Because m is tiny the particle effectively has no net force on it, so the drag force must cancel
out the other forces. From this we can find velocity.\\

\subsection{What is $\tau$?}
Tau is a time constant that describes the behavior of a system with respect to time. By looking at
the motion equations for Brownian systems in Section (\ref{sec:bmotion-derivation}), we see that the
size of t with respect to $\tau$ decides what type of motion the particle will feel. On the ``small''
timescale the system displays free particle motion, but on the ``large'' timescale, the system
displays Brownian motion. $\tau$ can be thought of as the time the system switches from free to
Brownian motion.\\

\section{Finding a reasonable dt}
The timescale of our simulation 

\section{Equipartition Theorem for quadratic energies}
\begin{align*}
  Z &= \int_{-\infty}^{\infty} e^{-\beta Ax^2}dx = \sqrt{\frac{\pi}{\beta A}}\\
  <E> &= - \frac{d\ln Z}{d\beta} = -\left(\sqrt{\frac{\beta A}{\pi}}\right)\left(\frac{\sqrt{\beta A}}{2\sqrt{\pi}}\right)\frac{-\pi}{A\beta^2}\\
  &= \frac{k_BT}{2}\\
\end{align*}

\end{document}
