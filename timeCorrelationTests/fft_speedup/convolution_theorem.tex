\documentclass[11pt]{article}

\usepackage{amsmath, amsthm, amssymb} 

\newcommand{\N}{\mathbb{N}}
\newcommand{\Z}{\mathbb{Z}}
\newcommand{\Real}{mathbb{R}}

\begin{document}
	
\title{Convolution Theorem for Autocorrelation Speedup} 
\author{John Waczak} 
\date{\today}
\maketitle

\paragraph{Motivation} 	
	In order to validate the $dt$ usage in our dynein simulation, I have run multiple simulations for the bound and unbound states (by choice of parameters) at different values of $dt$. To compare the simulations, an autocorrelation function was generated for the potential energy data by dynein domain (tail, motor, binding). This function is of the form: 
	
	\begin{align}
		\rho(\tau) &= \frac{\int_{-\infty}^{\infty} [(f(t)-\mu)(f(t+\tau)-\mu)]dt}
		{\int_{-\infty}^{\infty}(f(t)-\mu)^2}dt \\ 
		\rho(k) &= \frac{\sum_{0}^{t_{max}-k} (f(t)-\mu)(f(t+k)-\mu)}
		{\sum_{0}^{t_{max}-k}(f(t)-\mu)^2}
	\end{align}
	
	Where (1) is the continuous version and (2) the discrete. The problem with this formula though is that it is slow. For a data set of n values we want to compute n different k's in order to generate our autocorrelation function. This is of order $n^2$ computations. 

\end{document}
